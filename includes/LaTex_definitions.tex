%%%
%
% haeufig benoetigte Festlegungen
%
%%%

%% Erweiterungspakete werden geladen

% erlaubt direkte Verwendung von Umlauten im Quelltext (latin1)
%\usepackage{umlaut} 
\usepackage[utf8]{inputenc}% http://ctan.org/pkg/inputenc
\DeclareUnicodeCharacter{20AC}{\euro}

% länderspezifische Einstellungen
% sorgt u.a. für die korrekte Silbentrennung,
% deutsche Bezeichner ("Tabelle", "Abbildung" etc.)
\usepackage[english]{babel}
\sloppy %Blocksatz erzwingen

% fuer deutsche W"orter in der Literaturliste
\usepackage{bibgerm}  

% Sonderzeichen, z.B. Euro mit \texteuro   
\usepackage{textcomp}         

% erlaubt einfache Auswahl der Art der Nummerierung bei Aufzählungen
\usepackage{enumerate}

% erlaubt das Einbinden von Grafiken
\usepackage{graphicx}

% erlaubt die einfache Aenderung der Seitenraennder
%\usepackage[left=15mm,right=15mm,top=19mm,bottom=19mm]{geometry}

\usepackage{tabularx}

\usepackage{booktabs}

\usepackage{pgfplots}
\pgfplotsset{compat=1.12}

\usepackage{algorithmicx}

\usepackage{tabu}

\usepackage{setspace} 

\usepackage{amssymb}

\usepackage{lastpage}

\usepackage{multicol}

\usepackage{pgf-pie,etoolbox}

\usepackage{wrapfig}

\usepackage{float}

\usepackage{longtable}

\usepackage{titleps}

\usepackage{morefloats}

\usepackage{eurosym}

\usepackage{ifthen}

\usepackage{textcomp}

\usepackage{epstopdf}

\usepackage{wrapfig}

\usepackage{colortbl}

\usepackage{listings}

\usepackage{lmodern}

\usepackage[backend=bibtex,style=authortitle-comp]{biblatex} 

\usepackage{setspace}
\onehalfspacing

\usepackage{chngcntr}
\counterwithout{footnote}{chapter}

\definecolor{light-gray}{gray}{0.95}

% -------------------------- Lists as Sections ----------------------------------

\makeatletter
\renewcommand\listoftables{%
    \section*{\listtablename}%
    \addcontentsline{toc}{section}{\listtablename}
    \@starttoc{lot}%
}
\renewcommand\listoffigures{%
    \section*{\listfigurename}
    \addcontentsline{toc}{section}{\listfigurename}
    \@starttoc{lof}
}
\renewcommand\lstlistoflistings{
    \section*{List of Listings}
    \addcontentsline{toc}{section}{List of Listings}
    \@starttoc{lol}
}
\makeatother

% ------------------------------ Listings --------------------------------------

\lstset{numbers=left, 
		numberstyle=\ttfamily, 
		breaklines=true,
		backgroundcolor=\color{light-gray},
		xleftmargin=.25in,
		xrightmargin=.25in,
		basicstyle=\ttfamily\scriptsize,
		keywordstyle=\color{blue}\ttfamily,
		stringstyle=\color{red}\ttfamily,
		commentstyle=\color{gray}\ttfamily,
		captionpos=b,
		language=[R/3 6.10]ABAP,
}

\renewcommand\lstlistlistingname{List of Listings}

% ------------------------------ Geometrie--------------------------------------
\usepackage{geometry,blindtext}
\geometry{a4paper,left=25mm,right=25mm, top=25mm, bottom=25mm, includeheadfoot}

% ---------------------------- Page Styles -------------------------------------
\renewcommand{\chaptermark}[1]{\markboth{#1}{}}

\usepackage{fancyhdr}

\fancypagestyle{fancyStyle}{
	\fancyhf{}
	\fancyhead[L]{\chaptername~\thechapter}
	\fancyhead[R]{\leftmark}
	\renewcommand{\headrulewidth}{0.4pt}% Line at the header visible
	\fancyfoot[L]{DHBW Mannheim}
	\fancyfoot[R]{\thepage} %pagenumber
	\renewcommand{\footrulewidth}{0.4pt}% Line at the footer visible
}

\fancypagestyle{plain}{
  	\fancyhf{}
  	\renewcommand{\headrulewidth}{0pt}% Line at the header invisible
  	\renewcommand{\footrulewidth}{0.4pt}% Line at the footer visible
	\fancyfoot[L]{DHBW Mannheim}
	\fancyfoot[R]{\thepage} %page number
}

% ----------------------------- Hyperlinks -------------------------------------
%\usepackage{breakurl}         % Zeilenumbruch f"ur URLs

% Links zum Anklicken im DVI- und PDF-Dokument
\usepackage{hyperref} 
\hypersetup{colorlinks
  ,linkcolor=black             % toc, Glossar-Begriffe, Seitenzahlen in Index und Glossar
  ,urlcolor=black              % URLs, die mit \url{} erzeugt wurden
  ,citecolor=black             % Literatur-Zitate, die mit \cite erzeugt wurden
  ,filecolor=red              % Verweise auf Dateien, hier nicht verwendet
  ,breaklinks=true            % Zeilenumbruch f"ur Links
  ,linktocpage                % Nur Seitenzahlen sind Links, nicht ganze Zeilen
}
\def\UrlFont{\sffamily} 

% ------------------------------ Glossar ---------------------------------------
\usepackage[
% nonumberlist,               % keine Seitenzahlen anzeigen
acronym,                      % ein Abk"urzungsverzeichnis erstellen
toc,                          % Eintr"age im Inhaltsverzeichnis
section                       % im Inhaltsverzeichnis auf Section-Ebene erscheinen
]{glossaries}                 % definiert den Befehl \printglossary
\renewcommand*{\glspostdescription}{} %Den Punkt am Ende jeder Beschreibung deaktivieren
 
%Ein eigenes Symbolverzeichnis erstellen
\newglossary[slg]{symbolslist}{syi}{syg}{Symbolverzeichnis}

%Glossar-Befehle anschalten
\makeglossaries
 
% ----------------------------- Stichwortverzeichnis ---------------------------
\usepackage{makeidx}          % definiert den Befehl \printindex
\makeindex                    % erzeugt fuenftes.idx f"ur den Index

% ----------------------------- Aussehen einer Seite ---------------------------
%\textheight240mm              % Hoehe des Textes
%\textwidth150mm               % Breite des Textes
%\topmargin-20mm               % oberer Rand
%\oddsidemargin-7mm            % linker Rand bei ungeraden Seitenzahlen
%\evensidemargin-7mm           % linker Rand bei geraden Seitenzahlen
%\pagestyle{plain}             % plain    = Seitenzahlen, aber keine Kopfzeilen
                              % empty    = ohne Seitenzahlen
                              % headings = mit Kopfzeilen
%\parindent0mm                 % kein Einr"ucken am Anfang eines Absatzes


% kein "haengender" Einzug der ersten Zeilen eines Absatzes
\setlength{\parindent}{0cm}

% vertikaler Abstand zwischen Absätzen
% (1ex entspricht der Höhe des Buchstabens x, diese Angabe ist
% relativ zur gewählten Schriftgröße und passt sich somit bei
% einer Änderung entsprechend an)
\setlength{\parskip}{1ex}

\newlength{\myx} % Variable zum Speichern der Bildbreite
\newlength{\myy} % Variable zum Speichern der Bildhöhe
\newcommand\includegraphicstotab[2][\relax]{%
% Abspeichern der Bildabmessungen
\settowidth{\myx}{\includegraphics[{#1}]{#2}}%
\settoheight{\myy}{\includegraphics[{#1}]{#2}}%
% das eigentliche Einfügen
\parbox[c][1.1\myy][c]{\myx}{%
\includegraphics[{#1}]{#2}}%
}

\makeatletter
\renewcommand{\@chapapp}{}% Not necessary...
\newenvironment{chapquote}[2][2em]
  {\setlength{\@tempdima}{#1}%
   \def\chapquote@author{#2}%
   \parshape 1 \@tempdima \dimexpr\textwidth-2\@tempdima\relax%
   \itshape}
  {\par\normalfont\hfill--\ \chapquote@author\hspace*{\@tempdima}\par\bigskip}
\makeatother

% ----------------------------------- Colors --------------------------------
\definecolor{custRed}{RGB}{250,117,107}
\definecolor{custGreen}{RGB}{169,250,107}

% Penalties
\clubpenalty = 10000
\widowpenalty = 10000
\displaywidowpenalty = 10000