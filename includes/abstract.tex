\vspace{3cm}
\begingroup
\centering{\Large \textbf{Abstract}}\\
\endgroup
\vspace{1.5cm}

Many hours are wasted in front of red traffic lights in a life time. Every commuter knows this. And with the hours wasted, money is lost and green house gases are emitted. Many places still deploy very simple constant signal control systems, unable to change and adapt to ever changing conditions. While adaptive control systems do exist, only few cities are able to deploy them due to their cost and sensor requirements. In the era of the Internet of Things and Big Data, it is possible to have a fully connected signal network, where every signal can base its switching decisions on the knowledge of the entire network.

This paper proposes an approach to accomplish this goal using data driven decision making processes while keeping the number of data sources to a minimum.


\vspace{1.5cm}
\begingroup
\centering{\Large \textbf{Kurzzusammenfassung}}\\
\endgroup
\vspace{1.5cm}

Innerhalb einer Lebenszeit werden viele Stunden vor roten Ampeln verschwendet. Jeder Pendler weiß das. Mit den verschwendeten Stunden wird Geld verloren und Treibhausgas ausgestoßen. Viele Orte setzen noch immer einfache konstante Signalsteuerungssysteme ein, die nicht in der Lage sind, sich an ändernde Bedingungen anzupassen. Es existieren zwar adaptive Steuersysteme, aber wegen der hohen Kosten und Sensorikanforderungen können nur wenige Städte diese einsetzen. In der Ära des Internet der Dinge und Big Data ist ein vollständig verbundenes Ampelnetzwerk möglich, in dem jede Ampel seine Entscheidungen aus dem Wissen des gesamten Netzwerks bilden kann.

Diese Arbeit schlägt einen Ansatz vor, mit dem dieses Ziel mittels datengetriebener Entscheidungsprozesse erreicht werden kann und dabei trotzdem die Anzahl der Datenquellen minimal gehalten werden.
