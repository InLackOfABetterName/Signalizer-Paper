\section{Motivation and Goal}

Every city faces the challenge of street and traffic planning, and every citizen knows the struggle of driving through the city during rush-hour. Green waves, especially long lasting ones, are a rare phenomenon and it usually tends to be a waiting game. Kilometer long jams and completely congested junctions are nothing unusual and it often takes hours for them to resolve.

This is not only stressful for the commuters stuck in traffic but has a serious impact on the environment as well.
With increasing wait times not only decreases the driver comfort, but green-house emissions increase just as much. Only very few people own cars capable of automatically turning off the engine when idling, so the engines tend to keep running. Even fewer people own hybrid or fully electric cars due to the still high initial costs and still uncommon charing stations, particularly in countrysides.

Even if the roads are not congested, a large portion of the travel time in cities comes from signal controlled junctions. Probably everyone knows the situation of waiting at a completely empty crossing during the night, only because the traffic signal is not switching to green.
This occurs when the signal plans are defined completely statically.

In times of very cheap and reliable electronics, the Internet of Things and infrastructures capable of handling petabytes of data, statically defined signal schedules seem obsolete.
With increasing amounts of sensors, more and more accurate data becomes available for planners and the plan quality improves, but static plans can only be good in common situations that are known to the planner.

A good schedule minimizes the waiting time (and thus unnecessary emissions and stress) while maximizing the throughput. Static schedules will inevitably make wrong assumptions when unusual events occur (a big soccer match, a redirection due to road maintenance or an accident).

That is where adaptive signal controllers come in. Adaptive control systems will use the data collected by sensors and reevaluate their schedules throughout the day in certain time intervals.
Cities currently already use sensors at traffic lights like induction loops.
These sensors are mostly used to be able to skip lanes where no cars are waiting. For cost reasons (both initial cost as well as maintenance costs) sensors cannot realistically be placed at every lane on every junction.

A smart adaptive control system should use the limited amount of sensors and maximize the use of the data collected by them. Data collected by the entire city network consisting of a relatively small number of strategically placed sensors could provide enough information for making accurate predictions of traffic flow to adjust signal control schedules based on expected traffic.

Such a system could not only provide decent switching schedules on junctions without any sensors at all, but could easily be improved by additional sensors to enhance predictions.

\newpage

\section{Thesis Structure}

After this introduction, this thesis starts with the preliminaries, the knowledge foundation for the rest of the thesis. A short part about traffic signal guidelines and commuter stress is followed by a part about traffic light sensors, what types there are and how they work, and one about artificial neural networks.

Thereafter, an analysis of the current state of signal switching reveals more detailed goals for this thesis, which are then conceptually solved in chapter \ref{concept}. The architecture of the whole traffic light control system developed in this thesis and the details of its components are established.

Following the concept are some implementation details of the traffic light control system and a presentation of the result with a validation of the performance of the implementation against different other traffic light control mechanisms.

A critical discussion of all the previous and an outlook to further improvements and developments finishes the thesis as the last chapter.