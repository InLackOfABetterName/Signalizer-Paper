\section{Motivation and Goal}

Cool quote here?

Every city faces the challenge of street and traffic planning and every citizen knows the struggle of driving through the city during rush-hour. Green waves, especially long ones, are a rare phenomenon and it usually tends to be a waiting game.
With increasing wait times not only decreases the driver's happiness, but also green-house emissions increase as only very few people own cars that automatically turn off while standing. Even fewer own hybrid or fully electric cars.

During less congested hours it also not unlikely to wait in front of red lights for minutes.

These problems can be improved by implementing good traffic light schedules. But what is a \emph{good} schedule?

A good schedule minimizes the waiting time (and thus emissions) while maximizing the throughput. This not only the common predictable way, but also when unexpected sudden high traffic situations arise.

- There need to be a lot more about the idea here. We need to establish our idea before going into the preliminaries. Else the reader wont have a context when reading the preliminaries. Maybe we even have to do an extra section or chapter for the idea because it is not trivial and we should explain it well.

Cities currently already use sensors at traffic lights like induction spools to improve the traffic flow.\footnote{soure needed} As the traffic sensors can hardly be placed at every traffic light the idea of this thesis is to make a prediction for the road crossings where no sensors are present. Therefore one to multiple neural networks get fed with the traffic sensor data and predict where traffic goes and which traffic lights will receive a high load in the near future. With this information the prototype developed should calculate how to best switch the traffic lights to achieve near optimal throughput and wait times.

\section{Thesis Structure}

- Write this at the end, because only then the structure is set in stone.