This chapter will discuss the previously presented results of the research and provides and outlook of what further possibilities there are.

A comparison with other adaptive traffic control systems (for example one of those listed in \autoref{atcs_overview}) was not possible, as these systems are generally not freely available. The results shown in the previous diagrams have thus been compared to statically controlled signals with unoptimized schedules. While this seems like an easy and unrealistic comparison to win, it is actually not quite as unrealistic and as such still very useful. Many cities put afford into optimizing their streets  that carry most of the traffic during rush hour and this schedules usually work well. However when going off of these streets, the situation can change entirely. Suddenly, drivers are confronted with seemingly random switching schedules without any sensor input or interaction between intersections.

The results, especially those in shown \autoref{stress_based_results} and \autoref{agent_statistic_learned}, show that the proposed approach to signal controlling can in theory yield significant improvements over the static schedules, given a high prediction precision. The currently implemented neural networks have not yet shown the required precision which yield these results. The stress based controlling mechanism has shown to be an improvement, however not by the large margin that was initially expected. Where the low prediction precision comes from is not entirely known. It might be caused by a wrong training method are an insufficient error correction. These topics can be material for additional studies.

Compared to other controlling systems, this approach also does not explicitly support green waves, neither automatically nor manually. In fact, the proposed control system does not take any manual input at all. Additional networking concepts would need to be introduced and integrated with the signal prioritization in order to form green waves that propagate to the correct intersections with the correct order and timing, for green waves to be possible in this system. This is another field for optimization research in future projects.

The presented simulation has been able to proof the concept. While the implementation is not yet able to present an immediate economic benefit for cities, the potential can clearly be seen. The infrastructure required is already in place, so a migration to such a system could be done with a fairly low overhead, without roadwork to place additional expensive sensors and without expensive traffic analysts calculating signal schedules.

The case with perfect prediction precision has also shown how this system can reduce the vehicle buildup and prevent traffic jams, which could improve green house emissions and fuel consumptions, which is environmentally beneficial to both the city and the citizens. This is an area that has not seen much research.

Overall the results are promising and show, that traffic signal controlling is not a solved problem.