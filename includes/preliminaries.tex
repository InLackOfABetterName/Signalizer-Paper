\section{Traffic Signal Guidelines}

Different countries have different laws and guidelines that control how traffic signals are built and also how they are controlled.
In Germany for example there is the ''Richtlinien für Lichtsignalanlagen`` (in english Guidelines for Light Signal Systems), short RiLSA. Relevant points are guidelines for the length of the signal phases and proposals on how to structure model driven control systems. These use a model of the world (usually the street network) and use that to plan their switching schedule. The model is then continuously reevaluated to improve or change the schedule.

Other important guidelines, that are defined in these documents, are for example which lanes of an intersection actually require signaling, in which order which signal lights will activate and the basic design of the devices them selves, which includes the design of the walking figure of pedestrian signals. Of course, these guidelines also define a common vocabulary, so communicating information about signal systems becomes easier.

The German RiLSA for example defines a signal group as a group of one or more signals that control certain traffic flows together and will always show the same signal. It also defines the term ''phase`` as the part of a signal program during which the signal state of the system stays the same. \cite{rilsa}

By the definition of a phase from the RiLSA, this part of the signal program will be active for a certain time frame, and as such the term will be used in the sense of a duration for a signal state in the rest of the paper.

\section{Commuter Stress}
\label{commuter_stress}

Commuting to work is generally not a great activity, especially when the trip tends to be longer. Unexpected events and delays are usually stressful. Driver stress strongly correlates with time urgency as shown by the research in \cite{aggressive_behaviour}. Waiting in front of red lights might not necessarily be unexpected, but it is still stressful when the clock is ticking in your mind. In these situations stressed drivers are tempted to ignore the red signal and drive anyway, risking a severe accident. Junctions blocked by a heavily congested road will finally lead to nobody being able to move, whether the lights show green or not, which is very frustrating.

\section{Traffic Sensors}
\label{trafficSensors}

Central to signal scheduling are statistics about the traffic flowing through the junctions. A common statistic that is usually collected is the number of cars that pass by in a certain time frame, also called traffic count. This statistic is so simple to collect, that it is still often done manually by people.

In urban areas however, this is becoming unpopular as automatic sensors are increasingly common. There are quite a few different types of automatic traffic sensors with different properties. These sensors can generally be divided into two groups: In-pavement sensors and over-roadway sensors. \cite{tmg}

\begin{figure}[ht]
	\centering
	\includegraphics[width=8cm]{figures/induction-loop}
	\caption[Visualization of an induction loop]{Visualization of an induction loop\protect\footnotemark}
	\label{induction-loop}
\end{figure}


In-pavement sensors are directly embedded into the street. Common examples for these are the well known inductive loop (see \autoref{induction-loop}), pneumatic road tubes and piezo-electric sensors. Inductive loops measure the inductance in their electric loop, piezo-electric sensors and pneumatic tubes measure the pressure applied on the road, to detect cars. Depending on the implementation, these sensors can not only detect a vehicle, but also classify vehicles based on weight or size. The speed of passing vehicles can usually not be detected directly, but by combining multiple of these simple sensors, is is possible to estimate the speed of the vehicles.

\footnotetext{\url{http://s.hswstatic.com/gif/red-light-camera-loop.gif}}

\begin{table}[hb!]
	\centering
	\begin{tabular}{l|c|c|c|c}
		Name & Installation & Size & Weight & Speed \\
		\hline\hline
		Induction loops & in-pavement & no & no & no \\
		Piezoelectric sensors & in-pavement & no & yes & no \\
		Pneumatic road tubes & on-pavement & no & yes & no \\
		Laser sensors & over-road & yes & no & yes \\
		Light grids & over-road & yes & no & yes \\
		Radar sensors & over-road & yes & no & yes \\
		Visual sensors & over-road & yes & no & yes \\
	\end{tabular}
	\caption{Overview of traffic sensors}
	\label{sensor_overview}
\end{table}

The main advantage of these sensors is the cost and the simple installation. The obvious downside of in-pavement sensors is the fact that damaging the pavement will lead to damages to the sensors.

Over-road sensors are mounted over the street. Usual mounting points are existing traffic signs and traffic lights. The technologies used for these sensors vary in a wide range: Infrared, radar, laser, optical cameras, sound and more. They differ in precision and use cases. Most of them can measure the passage of vehicles, velocity, vehicle class, license plates and more. These sensors do not suffer from the problems as in-pavement sensors, but have a new set of problems. Sensors that work optically are often affected by bad weather conditions that introduce too much noise in the picture. The main advantage of these sensors is the amount of additional information that can be collected and their fast deployment. \cite{tdh}

\begin{figure}[ht]
	\centering
	\includegraphics[width=8cm]{figures/overroad-sensor}
	\caption[Possible over-road sensor setup]{Possible over-road sensor setup\protect\footnotemark}
	\label{overroad-sensor}
\end{figure}

\footnotetext{\url{http://www.aldridgetrafficcontrollers.com.au/Images/UserUploadedImages/112/Technical_drawing_Grey.jpg}}

\autoref{sensor_overview} shows common sensors and a few relevant properties.

Given the sensors are in place and generate electrical signals for passing vehicles, these signals need to be collected somewhere. In simple systems, these sensors feed right back into a local signal system. More advanced systems will collect the vehicle counts and other information in a central data center. Depending on the sensors the transmission is handled by the sensor itself or through additional devices. Modern sensors are directly connected to the Internet, using wireless broadband networks, which are very common and reliable in cities. An example setup at an crossing can be seen in \autoref{overroad-sensor}.

\section{Artificial Neural Networks}
\label{sec:ann}

``A neural network is a parallel distributed information processing structure in the
form of a directed graph''. \cite{introToNNs} The nodes of the graphs are called processing elements, with the links between them called connections. Each processing element can have any amount of input and output connections. But, the output connections of one processing element all have to carry the same value. Processing elements process the incoming data with a transfer function. \cite{introToNNs}

There is a separation of biological and artificial neural networks. Artificial neural networks (ANNs) try to model human congnition and biological neural networks with neurons. \cite{logicalCalculus1943} 
The big advantage of ANNs over algorithms that were explicitly developed for a task is that they can learn behavior like classification from a set of example data. 

\begin{figure}[ht]
	\centering
  \includegraphics[width=15cm]{figures/neuron_structure}
	\caption[Structure of a neuron]{Structure of a neuron \protect\footnotemark}
	\label{neuron}
\end{figure}

\footnotetext{\url{https://upload.wikimedia.org/wikipedia/commons/thumb/6/60/ArtificialNeuronModel_english.png/600px-ArtificialNeuronModel_english.png}}

Figure \ref{neuron} shows the structure of a neuron. A neuron has a set of $n$ input variables $x_1, x_2, x_3, ..., x_n$ which either are connected to the output of another neuron or the input to the neural network that is given to the network by the user. Each input variable is multiplied with a weight $w_1j, w_2j, w_3j, ..., w_nj$ where $j$ is the iteration of the weight. The products of the variables and their specific weights are then summed (transfer function) and passed to an activation function that determines what the output of the neuron is. \cite{introToNNs}

\begin{figure}[ht]
	\centering
  \includegraphics[width=11cm]{figures/activation_functions}
	\caption[Step and sigmoid activation function]{Step and sigmoid activation function \protect\footnotemark}
	\label{activation_functions}
\end{figure}

\footnotetext{\url{http://www.cs.nott.ac.uk/~pszgxk/courses/g5aiai/006neuralnetworks/images/actfn002.jpg}}

Examples for the activation function would be a step function that just compares the value to a threshold and puts out 0 or 1 or a sigmoid function that puts out a value between 0 and 1. A plot of both can be seen in figure \ref{activation_functions}. $t$ in the case of the step function is a threshold that the input value has to pass in order to activate the neuron. In most applications the step function is used because it can differentiate better between values that are around 0. The output can then be given to the connected neurons or the output of the neural network. \cite{introToNNs}

The process of designing an ANN to solve a problem is to first define input and output vectors and a network of neurons and connections that process the input vector and send the result to the output vector. Then a set of example data is passed through the ANN and a learning algorithm tries to find a configuration of the neurons weights to produce the desired outputs for the respective inputs. The ANNs form their own understanding of the problem while learning and can then, with enough training, solve problems that were not learned beforehand.

An important aspect of ANNs is how the neurons are interconnected. The most basic structure an ANN can have is that of feedforward neural networks. In this model the ANN is divided into layers and every layer only passes its information to the next layer. Thus, the input is fed forward through the layers until it is at the output layer, without loops in the network. In contrast to that, there are recurrent neural networks that allow the graph of neurons to have cycles. This allows the network to have some kind of state and keep values to use when processing data at a later point in time. An example for the use of recurrent neural networks would be a word processor that takes single words but can process whole sentences by memorizing what words were already passed through. \cite{nnsAndPatternRecog} There are a lot of sub categories of these structures and other architectures of ANNs, but feedforward and recurrent is the most important differentiation and the most relevant for this thesis.\cite{introToNNs}

\begin{figure}[ht]
	\centering
  \includegraphics[width=10cm]{figures/perceptron}
	\caption[Structure of a perceptron]{Structure of a perceptron \protect\footnotemark}
	\label{perceptron}
\end{figure}

\footnotetext{\url{http://www.nnwj.de/uploads/pics/1_2-perceptron.gif}}

Feedforward neural networks that only have an input layer and an output layer are called perceptrons. Such a network is shown in figure \ref{perceptron} They are the simplest kind of ANN, but they are very limited in their use. Because a neuron can only distinguish between two categories, a neural network with only two layers is not even able to map the logical XOR (exclusive or) operation. \cite{Minsky1988} This is also known as the problem of linear separability. The advantage of this kind of neural network is that it does not need a lot of learning data and can use a really simple learning algorithm called the delta rule.

The delta rule is probably one of the most simple learning rules and will in this thesis be used to explain the general idea of learning rules. It calculates a weight delta $\Delta w_{ji}$ for every $i$-th weight of neuron $j$ like this:

\begin{center}
	$\Delta w_{ji} = \alpha (t_j  y_j)g'(h_j)x_i$
\end{center}

With $\alpha$ being the learning rate, which directs how fast the network is reacting to trends, $t_j$ the target output, $y_j$ the actual output, $g$ being the activation function, $h_j$ the sum of the weighted inputs and $x_i$ the $i$-th input. This weight delta is then applied to the actual weight of the input. This works, because the higher the difference between the target output of the neuron and the actual input, the higher the impact on the weight and the sign of this difference also defines in which direction the weight changes. The derivation of the activation function shows high values when the sum of the weighted inputs is near the threshold, thus the neuron is uncertain and a slight change in the weights could adjust the decision in one or the other direction. Furthermore the higher the input value the higher the change in the weight because it impacts the output more than lower input values. \cite{deltaRule}

\begin{figure}[ht]
	\centering
  \includegraphics[width=8cm]{figures/multilayer_XOR}
	\caption[Structure of a multilayer perceptron that implements an XOR]{Structure of a multilayer perceptron that implements an XOR \protect\footnotemark}
	\label{multilayer_XOR}
\end{figure}

\footnotetext{\url{https://upload.wikimedia.org/wikipedia/commons/7/7b/XOR_perceptron_net.png}}

A bit more complex than perceptrons are multilayer perceptrons. They not only have an input and an ouput layer, but also one to several hidden layers that get fed from the previous layer and feed into the following layer. With this architecture more complex problems like XOR can be solved. The structure and weights of an XOR neural network are shown in \autoref{multilayer_XOR}. \cite{introToNNs}

Above was already shown how learning in simple ANNs takes place, but besides the function that determines how much a weight of a neuron changes, ANNs are also divided by the form of learning. This depends on how the learning set given to the neural network is created and composed. In the following the three most popular are described, being supervised learning, unsupervised learning and reinforcement learning. \cite{introToNNs}

With supervised learning, the learning set consists of pairs of input and output vectors and it thrives to adjust the weights of the neural network so that it maps the function described by the learning set. Then there is unsupervised learning which just takes for example a cost function, that calculates some costs from the input and output of the ANN, and a learning set, only consisting of input data, and tries to minimize the cost function by adjusting the weights. The third possible scenario is reinforcement learning. This happens when there is no data set to learn from, but an agent interacts with some world and learns which interactions are good and which are bad by observing the effects. This is for example used when an artificial intelligence that plays chess, is playing against itself to get better. \cite{introToNNs}

A problem with artificial neural networks is that the higher the complexity of the input and output vectors get, the higher is the demand for more learning data to get good results when using it. This is especially problematic for supervised learning algorithms, because then the data has to be tagged by humans. This problem can be somewhat solved with reinforcement learning because it does not need the huge amounts of input from a human who is slow. This has to be taken into account when weighing between detail of the input and output and the amount of learning data needed.