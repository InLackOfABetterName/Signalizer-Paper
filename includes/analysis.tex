This chapter will give an insight into the problems involved with traffic light control systems, what different solutions exist and how the problems are solved there. Based on this information the requirements for the new solution will be established.

=== am ende quasi requirements (ist, soll)===

Traffic light control systems have basically three main problems to solve:

\begin{enumerate}
	\item Which signals to switch
	\item When to switch signals
	\item How long signals should stay green
\end{enumerate}

Which signals to switch seems to have an obvious answer: All signals red except the street that is allowed to drive. In the real world however, the answer is not that simple. In order to maximize the throughput as many streets as possible should allow vehicles to drive. Considering traffic priority other rules makes this a complex problem. In the best case, all signals would be green and thus making traffic signals obsolete as seen in free-flow highway interchanges, because there the streets do not cross each other on the same level. In intersections save driving must be guaranteed by the signal control system. So there is a trade off required between throughput (all signals green / no signals at all) and driver safety (only one green signal at any given time).

\begin{figure}[ht]
	\centering
	\includegraphics[width=8cm]{figures/unrestricted_intersection}
	\caption{An unrestricted intersection}
	\label{unrestricted_intersection}
\end{figure}

In an unrestricted intersection of two streets (see \autoref{unrestricted_intersection}), a vehicle can enter from one of four directions and can leave the intersection in any of the three remaining directions (ignoring u-turns). That makes $4 \times 3 = 12$ different traffic lights, if every way through the intersection would be signaled separately, which is the case in larger intersections. All possible groups of signals would be given by the power set of the twelve different signals, which gives $2^{12} = 4096$ different groups. Finding the desired set signal groups can be modeled as a constraint satisfaction problem which grows exponentially with the number of signals in the intersection.

The second problem, deciding when to switch a certain signal group, is the main problem of this paper. Signal switching generally introduces overhead and reduces the throughput, so the amount of switches should be minimized. At the same time having to long phases will increase the latency (driver wait time) for other streets. This is also part of the third problem, how much time each phase gets. This is an optimization problem between maximizing the throughput on congested streets while still keeping the waiting times on other streets on an acceptable level. This is especially hard if two congested streets intersect.

Traffic lights can be controlled in one of two ways at the highest level: Statically or adaptive.

\paragraph{Static} Statically controlled traffic lights use predefined switching plans that are usually planned by a city's traffic management. In order to develop these switching plans, a lot of research has to be conducted about the expected traffic flow, including manually counting cars are installing traffic sensors. Early traffic lights have been build with their switch timings integrated into the control electronics, so the plans developed for the intersections have to be as good as possible as they cannot be changed easily. More modern static traffic control systems can be reprogrammed if necessary which can often be observed, when construction work is ongoing on a link adjacent to an intersection.

\paragraph{Adaptive} Adaptively controlled traffic lights are able to dynamically adapt (hence the name) to changing traffic flow. Some systems still relay on a set of static plans to choose from (for example the SCATS, Sydney Coordinated Adaptive Traffic System), while others produce their own timing plans. These adaptive systems all depend heavily on automatic traffic sensors, some require them at every line of an intersection (for example the SCOOT, Split Cycle Offset Optimisation Technique, system). The cost of the system and the required sensor coverage is one of the reasons, adaptive control systems are still not very widely adopted. The systems often work in cycles (rounds of signal phases) and optimize the cycle time and the split (which phase gets how much time). \cite{atcs_overview}.

Based in this information two main goals of this project:

\begin{enumerate}
	\item The system should be able to produce good switching plans with a small number of sensors (reduction of initial fixed cost)
	\item Require as little manual user interaction (traffic analysts changing plans or parameters) as possible (reduction of maintenance cost)
\end{enumerate}

In order to be able to work with with street network, that is not completely covered with traffic sensors, a mechanism to get a traffic estimation of the places where no sensors are. This could be done by utilizing additional data sources ranging from manually collected statistical data used for the traditional static planning to smartphone geo-locations of drivers, which leads into the area of swarm intelligence not covered by this paper. Another approach would be to use the available sensors for a prediction model, either pre-calculated or calculated on the fly.

Given a modeled street network, the system should be able gather all required information by itself. 


- Neural networks. Is there a pattern how to  generate the neural network? Should it grow on its own? What type of neural network do we need?

- Stress level. Is there already something like the stress level algorithm. Growth of stress level. Can set up a algorithm that builds the perfect stress level for certain inputs? On what level do we have to set up the stress level calculation? Maybe whole streets? What parameters go into the algorithm? Limits for stress levels for the switching of the lights. Can we stick to the idea of green waves with our stress level algorithm? Can we explicitly make green waves? Is our algorithm better than a static signal wiring, if we just calculate it with perfect predictions? How much worse does it get if the predictions are off? Can we make the algorithm even better after we did the calculation examples. Idea: We should probably split off our stress level calculator to be able to monitor other signal wirings. I think the stress level growth algorithm really has to be more or less perfect. It is our metric to determine later if our prototype is better than existing solutions. Maybe compare signal wirings that we know are better and worse with the stress level algorithm and show empirically that our algorithm is good. Does the stress level approach have disadvantages? Are there other metrics? Maybe this and the neural networks stuff have to be separate chapters? This would also make the titling of the chapters less generic, which I think is good and the work could be split up effectively. Maybe the system deciding on the signals that should be changed can not even find the optimal solution in reasonable time. The system is split up in two phases one that calculates stress levels and one that the signals. Maybe the calculation of the stress levels should even work with events in the end. When a traffic light switches from red to green it sends events to the next node and then the new stress level calculation for the traffic light is executed. This is too detailed for this chapter, but I think the idea is good.

- Exploration of different street intersections. Maybe in the sense of inputs for a neural network, because the input nodes and stuff should be generated by the application for every possible crossing. Are there especially challenging street crossings to implement? Does there have to be different stress level algorithms for different crossings?

