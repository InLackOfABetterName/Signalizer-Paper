This chapter will give an insight into the problems involved with traffic light control systems, what different solutions exist and how the problems are solved there. Based on this information the requirements for the new solution will be established.

Traffic light control systems have basically three main problems to solve:

\begin{enumerate}
	\item Which signals to switch
	\item When to switch signals
	\item How long signals should stay green
\end{enumerate}

Which signals to switch seems to have an obvious answer: All signals red except the street that is allowed to drive. In the real world however, the answer is not that simple. In order to maximize the throughput as many streets as possible should allow vehicles to drive. Considering traffic priority other rules makes this a complex problem. In the best case, all signals would be green and thus making traffic signals obsolete as seen in free-flow highway interchanges, because there the streets do not cross each other on the same level. In intersections save driving must be guaranteed by the signal control system. So there is a trade off required between throughput (all signals green / no signals at all) and driver safety (only one green signal at any given time). \cite{intersection_collision_avoidance}

\begin{figure}[ht]
	\centering
	\includegraphics[width=8cm]{figures/unrestricted_intersection}
	\caption{An unrestricted intersection}
	\label{unrestricted_intersection}
\end{figure}

In an unrestricted intersection of two streets (see \autoref{unrestricted_intersection}), a vehicle can enter from one of four directions and can leave the intersection in any of the three remaining directions (ignoring u-turns). That makes $4 \times 3 = 12$ different traffic lights, if every way through the intersection would be signaled separately, which is the case in larger intersections. All possible groups of signals would be given by the power set of the twelve different signals, which gives $2^{12} = 4096$ different groups. Finding the desired set signal groups can be modeled as a constraint satisfaction problem which grows exponentially with the number of signals in the intersection.

The second problem, deciding when to switch a certain signal group, is the main problem of this paper. Signal switching generally introduces overhead and reduces the throughput, so the amount of switches should be minimized. At the same time having to long phases will increase the latency (driver wait time) for other streets. This is also part of the third problem, how much time each phase gets. This is an optimization problem between maximizing the throughput on congested streets while still keeping the waiting times on other streets on an acceptable level. This is especially hard if two congested streets intersect.

Traffic lights can be controlled in one of two ways at the highest level: Statically or adaptive.

\paragraph{Static} Statically controlled traffic lights use predefined switching plans that are usually planned by a city's traffic management. In order to develop these switching plans, a lot of research has to be conducted about the expected traffic flow, including manually counting cars are installing traffic sensors. Early traffic lights have been build with their switch timings integrated into the control electronics, so the plans developed for the intersections have to be as good as possible as they cannot be changed easily. More modern static traffic control systems can be reprogrammed if necessary which can often be observed, when construction work is ongoing on a link adjacent to an intersection.

\paragraph{Adaptive} Adaptively controlled traffic lights are able to dynamically adapt (hence the name) to changing traffic flow. Some systems still relay on a set of static plans to choose from (for example the SCATS, Sydney Coordinated Adaptive Traffic System), while others produce their own timing plans. These adaptive systems all depend heavily on automatic traffic sensors, some require them at every line of an intersection (for example the SCOOT, Split Cycle Offset Optimisation Technique, system). The cost of the system and the required sensor coverage is one of the reasons, adaptive control systems are still not very widely adopted. The systems often work in cycles (rounds of signal phases) and optimize the cycle time and the split (which phase gets how much time). \autoref{atcs_overview} gives an overview over a few common commercial systems. \cite{atcs_overview}

\begin{table}[ht!]
	\centering
	\begin{longtabu}{l|cl|p{9cm}}
		\rowfont{\bfseries}
		Name & Year & Country & Features and Methodologies \\
		\hline\hline
		SCOOT & 1970 & UK & Optimizes Splits, Cycle and Offsets; real-time optimization of signal timing \\
		\hline
		SCATS & 1970 & Australia & Optimizes Splits, Cycle and Offsets; selects from a library of stored signal timing plans \\
		\hline
		OPAC & 1990 & USA & The network is divided into independent sub-networks \\
		\hline
		RHODES & 1990 & USA & Mainly for diamond interchange locations \\
		\hline
		ACS Lite & 1990 & USA & Operates with predetermined coordinated timing plans; automatically adjust splits and offsets accordingly \\
	\end{longtabu}
	\label{atcs_overview}
	\caption{Overview of commerical adaptive traffic control systems}
\end{table}

A problem that both static and adaptive control systems very often under heavy congestion have, is that already congested lanes are signaled green, even though no car can actually drive. In this situation the control system should skip the already congested lanes and distribute the time to other links. This may not reduce the overall congestion on the already congested road, but it will reduce the chance of vehicles standing inside the intersection and improves the latency for the free lanes.

Based in this information two main goals of this project:

\begin{enumerate}
	\item The system should be able to produce good switching plans with a small number of sensors (reduction of initial fixed cost)
	\item Require as little manual user interaction (traffic analysts changing plans or parameters) as possible (reduction of maintenance cost)
\end{enumerate}

In order to be able to work with with street network, that is not completely covered with traffic sensors, a mechanism to get a traffic estimation of the places where no sensors are. This could be done by utilizing additional data sources ranging from manually collected statistical data used for the traditional static planning to smartphone geo-locations of drivers, which leads into the area of swarm intelligence not covered by this paper. Another approach would be to use the available sensors for a prediction model, either pre-calculated or calculated on the fly.

Given a modeled street network, the system should be able gather all required information by itself. The methods developed should be applicable to any kind of street network, however tests are only done with the following intersection types (see \autoref{intersection_types}):

\begin{enumerate}
	\item Unrestricted intersections
	\item Unrestricted T-junctions
	\item Forks (one lane splits into two lanes)
	\item Merges (two lanes merge into one lane)
\end{enumerate}

There are many variations of these four basic intersection types that try to improve certain aspects of the intersections. When it comes to signal planning however, many of the variations have the same or very similar plans. Some design variations are better suited for intersections that are not fully signalized and thus rely on traffic priority rules, often accomplished by adding auxiliary lanes that avoid the traffic signals. However as this project is specifically about traffic lights, the concept of partially signalized intersections is not part of this paper and as such these designs are ignored as well.

\begin{figure}[ht]
	\centering
	\includegraphics[width=8cm]{figures/intersection_types}
	\caption{Basic tested types of intersections}
	\label{intersection_types}
\end{figure}

\subsection*{Pedestrians}

Traffic signal control systems also have to control signals for pedestrian traffic. Modeling pedestrian traffic increases the complexity of the system significantly. In most intersections, only a few pedestrians are crossing the street per time unit compared to vehicles. In these situations a naive approach to pedestrian signal controlling would be to link pedestrian signal phases to corresponding vehicle phases. In Germany, many pedestrian signals switch to green even though vehicles are allowed to cross, relying on priority rules and driver awareness. Some intersection use yellow flashing lights or other signals to increase awareness, but in most cases the only indication is the pedestrian traffic light itself. As a result, pedestrians are usually allows to go parallel to the main vehicle traffic that is allowed to drive, which works reasonably well. In places where pedestrians are indeed a major part of the intersection traffic, integrating them into the vehicle traffic will lead to more congestions. In these places, pedestrians (and in the same way cyclists) should be able to completely avoid the vehicle traffic, for example by implementing footbridges or underpasses. Underpasses can be built very space efficient and cities with subways they can be cost efficient as well, as existing tunnels may be (re)used. For these reasons, pedestrian traffic will not be ignored in the simulations

\subsection*{Railways}

Normal trains will usually not intersect streets of it is possible to prevent it. Bridges and tunnels are the tools used to do that. In dense cities with many tram stations, the only options would be to move to a subway system or a suspension railroad system, both alternatives (and especially the latter) are not very cost effective and are usually only deployed in large metropolises. Other cities will thus be confronted with tram traffic being part of the vehicle traffic. A city's intentions to become ''greener`` (lower environmental impact) heavily influences its preference of public transit over private transit. In green cities, trams and trains will always be preferred over cars at intersections, which is called train preemption. If the required space for additional lanes is available, the same principle is applied to buses. This greatly reduced travel times of commuters in public transport at the cost of the travel time of car drivers. This is often intended in order to make public transport more attractive.

A simple implementation, which is often seen in practice, is to pause the traffic signal cycle as soon a rail signal in front of the intersection is activated and switch all traffic lights to red. When the rail signal after the intersection is deactivated (the train has passed the intersection), the signal cycle is continued where it was before. In this implementation, the intersection will come to a complete stop for car traffic during the time the train takes to pass, but is is very simple to implement. Any traffic signal control system can implement this. \cite{mutcd}, \cite{ptsnrc}

More advanced approaches will try to minimize the red signals by only stopping the traffic, that conflicts with the active train track. Lanes that are parallel to the train tracks could thus be allows to drive without creating a security risk. This of course requires more information than the simple approach. The control systems obviously needs to know which lanes will be in conflict with the train track. Depending on the defined phases (signal groups) for normal operation, it might be necessary to define special phases for these situations, if the existing phases to not fit (optimally). If the phases are the same, the control system could simply skip phases with conflicting lanes while keeping other phases unchanged.

Intersections that do not deploy train preemption will have to fully integrate the train tracks into the cycle planning. The rest of the paper will ignore trains for the simulation and signal control. It is assumed that the preemption can be easily implemented in one of the mentioned ways.
